%%%%%%%%%%%%%%%%%%%%%%%%%%%%%%%%%%%%%%%%%
% Medium Length Graduate Curriculum Vitae
% LaTeX Template
% Version 1.1 (9/12/12)
%
% This template has been downloaded from:
% http://www.LaTeXTemplates.com
%
% Original author:
% Rensselaer Polytechnic Institute (http://www.rpi.edu/dept/arc/training/latex/resumes/)
%
% Important note:
% This template requires the res.cls file to be in the same directory as the
% .tex file. The res.cls file provides the resume style used for structuring the
% document.
%
%%%%%%%%%%%%%%%%%%%%%%%%%%%%%%%%%%%%%%%%%

%----------------------------------------------------------------------------------------
%	PACKAGES AND OTHER DOCUMENT CONFIGURATIONS
%----------------------------------------------------------------------------------------

\documentclass[margin, 10pt]{res} % Use the res.cls style, the font size can be changed to 11pt or 12pt here

\usepackage{helvet} % Default font is the helvetica postscript font
%\usepackage{newcent} % To change the default font to the new century schoolbook postscript font uncomment this line and comment the one above
 \usepackage[svgnames]{xcolor} % color names

\usepackage{hyperref}
\hypersetup{
    colorlinks=true,
    linkcolor=blue,
    filecolor=magenta,      
    urlcolor=SeaGreen,
}
\urlstyle{same}

\usepackage{etaremune}

\setlength{\textwidth}{5.1in} % Text width of the document


\begin{document}

%----------------------------------------------------------------------------------------
%	NAME AND ADDRESS SECTION
%----------------------------------------------------------------------------------------

\moveleft.5\hoffset\centerline{\large\bf DANIEL YUDI MIYAHARA NAKAMURA} % Your name at the top
 
\moveleft\hoffset\vbox{\hrule width\resumewidth height 1pt}\smallskip % Horizontal line after name; adjust line thickness by changing the '1pt'
 
%\moveleft.5\hoffset\centerline{Biologist, Data Scientist, and PhD student } % Your address
\moveleft.5\hoffset\centerline{University of São Paulo, SP, Brazil}
\moveleft.5\hoffset\centerline{Phone: +11-99667-8820 | E-mail: dani\_ymn@outlook.com}
\moveleft.5\hoffset\centerline{\href{https://scholar.google.com.br/citations?hl=pt-BR&user=c0W8Cm8AAAAJ&view_op=list_works&sortby=pubdate}{Google Scholar} | \href{https://www.researchgate.net/profile/Daniel-Nakamura}{Research Gate}}

%----------------------------------------------------------------------------------------

\begin{resume}
 
\section{EDUCATION}

\textbf{University of São Paulo} \hfill 2022--current \\
Ph.D. in Bioinformatics
\begin{itemize}
\item Thesis: Lost biodiversity in light of museum genomics: Contributions from historical DNA to the systematics of rare and extinct frogs
\item Advisor: Taran Grant.
\end{itemize} 

\textbf{University of São Paulo} \hfill 2017--2021 \\
B.S. and Licentiate's degree in Biology
\begin{itemize}
\item Thesis: Body size, habitat, and sexual selection affect call evolution in Cophomantini (Anura: Hylidae: Hylinae)
\item Advisor: Paulo Durães Pereira Pinheiro.
\end{itemize} 

%----------------------------------------------------------------------------------------

\section{PUBLICATIONS}
{\sl In preparation}
\begin{etaremune}
% 1
\item \textbf{Nakamura DYM}, Grant T (In prep.) \emph{RNODE}: an R application to compare shared clades between phylogenetic trees. To be submitted to \emph{Cladistics}. 
\end{etaremune}

{\sl Under review}
\begin{etaremune}
% 4
\item Escalona M, [...] \textbf{Nakamura DYM}, Maneyro R, Castroviejo-Fisher S. Allometric constraint predominates over the acoustic adaptation hypothesis in a radiation of Neotropical treefrogs. Submitted to \emph{Integrative Zoology}.
% 3
\item Pinheiro PDP, Dallacorte F, Thompson J, Comitti EJ, \textbf{Nakamura DYM}, Garcia PCA. Two new species of the \emph{Boana semiguttata} clade. Submitted to \emph{South American Journal of Herpetology}. 
% 2
\item \textbf{Nakamura DYM}, Lyra ML, Grant T. Procedures for obtaining tissue samples from amphibian and reptile specimens for museomics. Submitted to \emph{Herpetologia Brasileira}. 
% 1
\item \textbf{Nakamura DYM}, Pinheiro PDP, Lyra ML, Faivovich J, Grant T. Historical DNA places a potentially extinct gladiator frog in the phylogeny of the \emph{Boana pulchella} Group (Anura: Hylidae). Submitted to \emph{Herpetologica}. 
\end{etaremune}

{\sl Peer-reviewed}
\begin{etaremune}
% 3
\item \textbf{Nakamura DYM}, Koffler S, Mello MAR, Francoy TM (2024) Resin foraging interactions in stingless bees: an ecological synthesis using multilayer networks. \emph{Apidologie} \textbf{55}:34.  \href{https://doi.org/10.1007/s13592-024-01082-8}{DOI}
% 2
\item Albuquerque-Pinna J, Jeckel AM, \textbf{Nakamura DYM}, Bernarde PS, Kocheff S, Saporito RA, Grant T (2024) Defensive alkaloid variation and palatability in sympatric poison frogs. \emph{Chemoecology}. \emph{In Press}. \href{https://doi.org/10.1007/s00049-024-00402-9}{DOI} 
% 1
\item \textbf{Nakamura DYM}, Escalona M, Pinheiro PDP (2024) Body size, habitat, and sexual selection affect call evolution in Cophomantini treefrogs (Anura: Hylidae: Hylinae). \emph{Biological Journal of the Linnean Society}. \emph{In Press}. \href{https://doi.org/10.1093/biolinnean/blae036}{DOI}
\end{etaremune}

{\sl Books}
\begin{etaremune}
% 1
\item Koffler S, Ghilardi-Lopes NP, Francoy TM, Albertini B, Leocadio J, Barbiéri C, \textbf{Nakamura DYM}, Caldeira BS, Silva L, Yamamoto Y, Saraiva A. (2021) Projeto \emph{cidadãoasf} - Protocolo de monitoramento de atividade de voo em abelhas sem ferrão utilizando ciência cidadã. Brazil, Santo André, UFABC. 72p. ISBN: 978-6-55-719028-9
\end{etaremune}

%----------------------------------------------------------------------------------------

\section{AWARDS}
\begin{itemize}
% 2
\item Best undergraduate presentation \hfill 2021 \\
II Ciclo de webinars WDA-LA: Estudiantes, Wildlife Disease Association 
% 1
\item Best undergraduate presentation \hfill 2019 \\
XXVI Semana Científica Benjamin Eurico Malucelli, University of São Paulo 
\end{itemize}

%----------------------------------------------------------------------------------------

\section{RESEARCH \\FUNDING}

\begin{itemize}
% 4
\item Ph.D. scholarship, R\$202,528.80 \hfill August 2022--July 2028 \\
FAPESP (grant 22/02789-0)
% 3
\item Undergraduate scholarship, R\$9,000.00 \hfill January 2021--July 2022 \\
PUB-USP
% 2
\item Undergraduate scholarship, R\$8,348.40 \hfill August 2019--July 2020 \\
FAPESP (grant 19/11096-5)
% 1
\item Technical support to research, R\$4,800.00 \hfill August 2017--July 2018 \\
CNPq (grant 372343/2017-1)
\end{itemize}

%----------------------------------------------------------------------------------------

\section{CONGRESS \\ PRESENTATIONS}

\begin{etaremune}
% 5
\item \textbf{Nakamura DYM}, Grant T (2023) Support in phylogenetics: Do resampling metrics predict optimality-based support in parsimony analyses? XXII Workshop of the Graduate Program in Bioinformatics, University of São Paulo, Brazil.
% 3
\item Pereira JM, \textbf{Nakamura DYM}, Homma MHM, Grant T (2023) The advertisement calls of three species of three species of \emph{Brachycephalus} (Anura: Brachycephalidae) from southeastern Brazil. X Brazilian Congress of Herpetology, Federal University of Southern Bahia, Brazil. 
% 3
\item \textbf{Nakamura DYM}, Pinheiro PDP, Lyra ML, Faivovich J, Grant T (2023) Historical DNA places a  extinct gladiator frog in the phylogeny of the \emph{Boana pulchella} Group (Anura: Hylidae). X Brazilian Congress of Herpetology, Federal University of Southern Bahia, Brazil. 
% 2
\item \textbf{Nakamura DYM} (2021) Fibropapillomatosis in sea turtles and the spirorchiid parasites. II Ciclo de webminars, Latin America, Wildlife Disease Association.
% 1
\item \textbf{Nakamura DYM}, Zamana RR, Gattamorta MA, Matushima ER (2019) Fibropapillomatosis in sea turtles: are the parasites Spirorchiidae (Trematoda: Digenea) possible vectors of \textit{Chelonid alphaherpesvirus 5}? XXVI Semana Científica Benjamin Eurico Malucelli, University of São Paulo, Brazil.
\end{etaremune}

%----------------------------------------------------------------------------------------

\section{SKILLS}

\begin{itemize}
\item Bash/Linux (advanced)
\item R (intermediate)
\item Python (intermediate)
\item Git and Github
\item LaTeX
\item Microsoft Office
\end{itemize}

%----------------------------------------------------------------------------------------

\section{TEACHING}
{\sl Undergraduate teaching assistant}
\begin{itemize}
% 4
\item BIZ0448 - Animal architecture: Evolution of metazoan body plans \hfill 2024 \\
University of São Paulo
% 3
\item BIZ0212 - Vertebrates \hfill 2020, 2023 \\
University of São Paulo
% 2
\item BIZ0213 - Invertebrates \hfill 2021 \\
University of São Paulo
% 1
\item 0410117 - Philosophy for Biological Sciences \hfill 2019 \\
University of São Paulo
\end{itemize}

{\sl Invited lessons}
\begin{itemize}
% 1
\item BIZ5749 - Systematics and evolution of amphibians and reptiles \hfill 2022 \\
University of São Paulo
% 1
\item BIZ0440 - Herpetology \hfill 2022 \\
University of São Paulo
\end{itemize}

%----------------------------------------------------------------------------------------

\section{OUTREACH}
{\sl Organization of public events}
\begin{itemize}
%2
\item É o Bicho  na Remo! \hfill 2024 \\
University of São Paulo, São Paulo, Brazil
% 1
\item Meliponicultura e ciência cidadã \hfill 2022 \\
University of São Paulo (EACH-USP), São Paulo, Brazil
\end{itemize}

{\sl Environmental educator}
\begin{itemize}
% 2
\item Internship in the Tamar Environmental Monitoring Program \hfill 2020 \\
Project TAMAR, Ubatuba, Brazil
% 1
\item Monitor in the BioBlitz Program, Natural Geographic Education \hfill 2018 \\
Instituto Butantan, São Paulo, Brazil
\end{itemize}

{\sl College prep teacher}
\begin{itemize}
% 2
\item Prefeitura de Jandira, São Paulo, Brazil \hfill 2021--2022
% 1
\item Universidade Cidade de São Paulo (UNICID), São Paulo, Brazil \hfill 2017
\end{itemize}

\end{resume}
\end{document}
